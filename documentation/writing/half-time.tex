\documentclass{article}
\usepackage[utf8]{inputenc}
\title{Classification of Population Activity in Parkinson's Disease}
\author{Gustav Röhss, Míriam Vall}
\date{VT 2020}
\begin{document}
\maketitle

\section*{Abstract}
\section*{Sammanfattning}

\newpage
\tableofcontents

\newpage
\section{Introduction}
Something about usage of machine learning in relation to medicine and diagnosis.
Something about usage of data analysis in relation to medical datasets.

\subsection{Parkinson's disease}
A quick primer on what it is.
Some symptoms it causes.
Parkinson's in relation to the Basal Ganglia (shorter).
Focus on introduction of different regions, beta activity, and beta oscillations.
Possibly something about treatment methods (if based on similar phenomena).

\subsection{Dataset}
The dataset contents.
The relevance of the dataset in relation to Parkinson's.

\subsection{Purpose and research question}
Achieve better understanding of activity in Basal Ganglia in Parkinson's-affected brains.
Is beta activity synchronized, and if so, to what degree?
Can activity in different animals be clustered?

\newpage
\section{Background}
Something about concepts needign to be understood in order to understand methods.

\subsection{Parkinson's disease}
Parkinson's in relation to the Basal Ganglia (longer).
The relationship(s) between different regions as they are understood currently.
Ensure decent glossary (GP, STR, STN, ...).
We don't want to introduce any additional medical terminology past this point.

\subsection{Feature extraction}
High dimensional dataset. 
Reducing the amount of dimensions by selecting most relevant ones.

\subsection{Clustering}
Unsupervised learning.
Finding (mathematically) natural groups/groupings of data.

\newpage
\section{Methods}
Several methods employed in order to create novel model.

\subsection{Discrete Fourier transform}
Transform from temporal to spectral domain.
Short introduction; implicitly understood by reader. 
Refer to more extensive reading.

\subsection{k-Means}
Higher-dimensional centroid-based clustering.
Used on extracted features to find clusters.

\subsection{Model}
Glossary: channels, values, epochs, epoch groups.
Splitting of channel values into epochs, fourier transform usage.
Model parameters, normalization, epoch group size, dropout.
k-Means, cluster centers, weighted cluster centers.
Basically, describe the assembly line.

\subsection{Custom datasets}
Construction of custom datasets for evaluation with the model.

\subsubsection{Purpose}
Seeking "ground truth".
Synchronized data should be classified as such.
Unsynchronized should not.
Goal it to attempt to create a dataset similar to real dataset.
Similar both in terms of output and heuristically/medically based on known truths about LFP.

\subsubsection{Dataset types}
Table/description of datasets.

\newpage
\section{Results}
Might (will probably) require subsectioning.
Results for real datasets.
Results for dummy data.
Bringing it together.
Cool graphs.

\newpage
\section{Discussion}
Model has limitations, dummy data has limitations.
Nearly impossible to "pure-math" prove anything.
Pending further research best bet might be to attempt to reproduce synchronization phenomena.

\subsection{Synchronization}
Does it seem likely?

\subsection{Model limitations}
Nigh impossible to work with mathematically.
Somewhat possible to work with statistically.
May require much further research.
Time could be spent better on other models.

\subsection{Custom dataset limitations}
Simplicity, "distance" from real LFP.
Stochasticity influencing results.

\newpage
\section{Conclusion}
Synchronization proven, yay or nay?
Further research required, always.

\end{document}
